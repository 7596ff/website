\documentclass[12pt]{article}
    \usepackage[margin= 1in]{geometry}
    \usepackage{url}
    \usepackage{breakurl}
    \usepackage[authordate,backend=biber]{biblatex-chicago}

	\usepackage{step}
    \usepackage{parskip}
	\usepackage{setspace}

	\pagestyle{myheadings}
    \markboth{}{McCarthy}

    \setlength{\parindent}{2em}
    \setlength{\parskip}{1em}
    \setstretch{1.2}

    \addbibresource{final.bib}

    \title{Should we strive for rationality?\thanks{I am extremely appreciative of Dr. Jack Kwong's
    class for sustaining my passion for the analytic.}}
    \author{Cassandra McCarthy}
    \date{2020-05-03}
\begin{document}
\maketitle

In this paper, I will be reading and invoking Paul's decision theory from
\citetitle{paul_what_2015-1} in an analysis of McKinnon's\footnote{As of writing, Veronica Ivy has
changed her name from Rachel McKinnon, but since her work was published under the former name, I
will use the former name throughout this paper.} \citetitle{mckinnon_trans*formative_2015-1}, which
argues that Paul's model does not apply to gender transition. I will explain why I agree with
McKinnon, and finally, I will attempt to offer a model for making transformative decisions that is
compatible with Paul's decision theory. 

\section{Paul}

Paul's paper\footcite{paul_what_2015-1} is a fascinating one. Paul begins with a \textit{Scenario}
that many prospective parents experience: a couple sits down and talks about whether or not they
want to have a child. Paul's \textit{Scenario} boils down to two choices, for (they decide to have a
child) and against (they decide to remain childless). She mentions that many parents decide to have
children because they feel that ``having a child will help them to live a fuller, happier, and
somehow more complete life.'' 

Section 2 of the paper begins to describe Paul's normative decision theory: ``[if] we can glean
approximate values for our outcomes and apply the right decision theoretic rules, we can conform to
the ordinary standard for rational decision-making.'' In other words, for Paul, we try as we can to
make rational decisions with whatever information we have. The rest of the section focuses on trying
to simplify and boil down Paul's \textit{Scenario} into a ``set of fine-grained, exclusive and
exhaustive propositions.''

Paul now invokes Frank Jackson's Mary the Neuroscientist thought experiment, where a brilliant
scientist named Mary has been living her entire life inside a black-and-white box, unable to
experience any form of color. She knows what it should be like to see color, based on her
understanding of how the eye sees color, how the brain processes color, and so on. But for Paul,
Mary has never had the \textit{epistemically transformative experience} of actually seeing color,
and such, Mary is in an \textit{epistemically impoverished state.} Mary can not know exactly what it
is or will be like to experience the color red. And such, for Paul, someone trying to decide whether
or not to have a child is in a similar impoverished position, ``because [they] do not know what it
will be like to have a child of [their] very own.\footnote{I prefer to use they/them pronouns when
making examples. I have portrayed Paul's writing to the same effect, but with more inclusive
language.}

Paul also shows that such an epistemically transformative experience as having a child is
additionally \textit{personally transformative}, because it may change ``what it is like to be
you.'' She lists reasons why having a child can be both epistemically and personally transformative,
and then begins to show why such a choice presented in \textit{Scenario} can not be made rationally.
It is fair to say that rational decisions can not be made from incomplete information. In a
normative decision model, choices are made from incomplete information.  Since it is impossible
for Mary or a prospective parent to ``rationally determine the values of the relevant outcomes,''
for Paul, it is not possible to make a rational decision about such a transformative experience.
Paul stresses many times that it is ``impossible'' to make a rational choice. I agree, but I wonder
why there is such a focus on being rational and making rational decisions all the time. 

\section{McKinnon}

This paper\footcite{mckinnon_trans*formative_2015-1} is ``composed of two related projects, tied
together by considering trans experiences of gender transition viz. transformative experiences.''
For the purposes of my paper, however, I only want to focus on McKinnon's disagreement with Paul.
McKinnon explains Paul's account of transformative experiences. She argues that the prospect of
undergoing a gender transition is the paradigmatic example of a transformative experience, due to
how much of daily life is gendered. 

McKinnon's fundamental disagreement is that ``for many trans people contemplating gender transition,
they know the expected utility of not transitioning.''\footnote{In her paper, McKinnon uses two
different terms to refer to the trans* community: ``trans'' to refer to transsexual individuals, or
those who have or want gender-affirming surgery, and ``trans*'' to refer to the broad spectrum of
trans people. While I do not personally agree with this emerging convention, I will use it for the
purposes of this paper.} McKinnon adapts Paul's \textit{Scenario} to gender transition as such: two
choice are presented, to transition or not to transition, and two outcomes from each choice exist,
happiness or unhappiness, for a grand total of four outcomes: `transition-happy',
`transition-unhappy', `not transition-happy', and `not transition-unhappy'.

McKinnon points to the distressing rate of trans* suicide. ``In a number of studies, the percentage
of trans* people who have attempted suicide is 41\%. Trans* people can be faced with an
``all-consuming'' need to begin transition. And such, for many trans* people, the `not
transition-happy' outcome is effectively impossible.

I end my summary with an important example explained by McKinnon, quoted in full: ``Suppose I have
to place a bet with my life. I know I’ll lose if I bet on red. But I have an unknown non-zero chance
of winning some unknown amount by betting on black. If I care about winning, the only rational
choice is to bet on black. So I should do that, according to normative decision theory. The same is
true for many trans people contemplating transition.''

\section{The Illusion of Choice}

I agree with McKinnon to an extent. While it is possible for someone to look into the future and
imagine themself in the same emotional and physical state as they currently inhabit, Paul is right
that it is impossible for someone to see 18 years down the road and imagine themeslf without a
child, and such, it is impossible to see far into the future having not transitioned. Many trans*
people in fact report to not even be able to conceptulalize life past the present, within the gender
that they themselves, as well as society, ascribe to them. 

This impossibility can be seen as an illusion of choice. For example, under the current neoliberal
economic system existing in most western countries, every person is offered a choice of whether or
not to work, to hold a job. However, if one choses not to hold a job, they face poverty,
homelessness, starvation, and in extreme cases, death. This can be simplified and shown as a
`work-or-die' existence. If we apply this rationale to the prospect of transitioning, trans* people
could be faced with the false choice of `transition-or-die', or
`transition-or-stagnate'.\footnote{As McKinnon stated, this does not include all trans* people, but,
for my purposes, it is a significant enough amount to make this point.}

Another false choice is created by inequality. Many trans* people face prohibitively expensive
surgeries, or doctor's appointments likely not covered by insurance to acquire hormone replacement
therapy. They also face abuse or ignorance from peers or family members. In these cases, the choice
not to transition has been made for them. 

I also argue that one can not rationally choose to die. This leads me to agree with Paul, in that
one can not rationally choose to transition (medically or otherwise), and one can not rationally
choose to die. One can also not physically choose to transition, when prohibited by costs or
otherwise. 

\section{The Cult of Rationality}

As Paul and I have shown, it is impossible to make rational decisions about transformative
experiences. However, essentially, ``so what?'' So what if we as human beings can not be perfectly
rational in every decision? Since the Enlightenment, where reason and empiricism flourished,
humanity has been obsessed with maximising the pleasure and minimizing the pain through careful
planning and decisions. Rationality was also viewed as superior because it was in opposition to
emotional reactions to stimuli by women and children. But since it is impossible to be perfect in
this regard, why even strive for perfection?

I do not mean to appear negative in this quandary. Instead, I propose a virtuous approach where
rationality is the upper vice and emotionality is the lower. Somewhere between these seems like the
way towards the good life. Life is all about responding to events and stimuli that are surprising,
unwanted, exciting, upsetting, tragic, and blissful. Rationality can only get a decision-maker so
far, and where it struggles, emotionality back it up. 

Choices could still be made taking reason into account. As before, one faced with a decision would
approximate values for each outcome. But they would also listen to what their heart says about the
outcome. In deciding to transition, if someone was equally capable of either outcome, rather, they
had the financial, emotional, and social stability to undergo transition, they would be able to do
what ``feels'' the most right.\footnote{I am aware that personal anecdotes do not have a place in
philosophical writings, but if I may: this is quite the situation I was in when I realized I was
trans*. Being able to make the decision with my heart and my head felt correct, even if I was not
consciously using this model.} This method of decision making flourishes in a more equitable
society, one without artificial boundaries in the way of the path to child-rearing, transitioning,
or even college. Armed with an idealistic way of making decisions, the question remains: what is to
be done to allow everyone to take advantage of the method?

\printbibliography

\end{document}

